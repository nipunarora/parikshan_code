% THIS IS SIGPROC-SP.TEX - VERSION 3.1
% WORKS WITH V3.2SP OF ACM_PROC_ARTICLE-SP.CLS
% APRIL 2009
%
% It is an example file showing how to use the 'acm_proc_article-sp.cls' V3.2SP
% LaTeX2e document class file for Conference Proceedings submissions.
% ----------------------------------------------------------------------------------------------------------------
% This .tex file (and associated .cls V3.2SP) *DOES NOT* produce:
%       1) The Permission Statement
%       2) The Conference (location) Info information
%       3) The Copyright Line with ACM data
%       4) Page numbering
% ---------------------------------------------------------------------------------------------------------------
% It is an example which *does* use the .bib file (from which the .bbl file
% is produced).
% REMEMBER HOWEVER: After having produced the .bbl file,
% and prior to final submission,
% you need to 'insert'  your .bbl file into your source .tex file so as to provide
% ONE 'self-contained' source file.
%
% Questions regarding SIGS should be sent to
% Adrienne Griscti ---> griscti@acm.org
%
% Questions/suggestions regarding the guidelines, .tex and .cls files, etc. to
% Gerald Murray ---> murray@hq.acm.org
%
% For tracking purposes - this is V3.1SP - APRIL 2009

\documentclass{acm_proc_article-sp}

\begin{document}

\title{Modeling Time Windows for Sandbox Testing}
%\subtitle{[Extended Abstract]
%\titlenote{A full version of this paper is available as
%\textit{Author's Guide to Preparing ACM SIG Proceedings Using
%\LaTeX$2_\epsilon$\ and BibTeX} at
%\texttt{www.acm.org/eaddress.htm}}}
%
% You need the command \numberofauthors to handle the 'placement
% and alignment' of the authors beneath the title.
%
% For aesthetic reasons, we recommend 'three authors at a time'
% i.e. three 'name/affiliation blocks' be placed beneath the title.
%
% NOTE: You are NOT restricted in how many 'rows' of
% "name/affiliations" may appear. We just ask that you restrict
% the number of 'columns' to three.
%
% Because of the available 'opening page real-estate'
% we ask you to refrain from putting more than six authors
% (two rows with three columns) beneath the article title.
% More than six makes the first-page appear very cluttered indeed.
%
% Use the \alignauthor commands to handle the names
% and affiliations for an 'aesthetic maximum' of six authors.
% Add names, affiliations, addresses for
% the seventh etc. author(s) as the argument for the
% \additionalauthors command.
% These 'additional authors' will be output/set for you
% without further effort on your part as the last section in
% the body of your article BEFORE References or any Appendices.

\numberofauthors{1} %  in this sample file, there are a *total*
% of EIGHT authors. SIX appear on the 'first-page' (for formatting
% reasons) and the remaining two appear in the \additionalauthors section.
%
\author{
% You can go ahead and credit any number of authors here,
% e.g. one 'row of three' or two rows (consisting of one row of three
% and a second row of one, two or three).
%
% The command \alignauthor (no curly braces needed) should
% precede each author name, affiliation/snail-mail address and
% e-mail address. Additionally, tag each line of
% affiliation/address with \affaddr, and tag the
% e-mail address with \email.
%
% 1st. author
\alignauthor
Nipun Arora \\
     \affaddr{NEC Research Labs}\\
     \affaddr{Princeton, NJ, USA}\\
     \email{nipun@nec-labs.com}
}

%All Authors
\iffalse 
\numberofauthors{3}
\author{

\alignauthor

Nipun Arora\\

       \affaddr{NEC Research Labs}\\

       \affaddr{Princeton, NJ, USA}\\

       \email{nipun@nec-labs.com}

% 2nd. author

\alignauthor

Abhishek Sharma\\

       \affaddr{NEC Research Labs}\\

       \affaddr{Princeton, NJ, USA}\\

       \email{absharma@nec-labs.com}

% 3rd. author

\alignauthor Gail Kaiser\\

       \affaddr{Columbia University}\\

       \affaddr{New York, NY, USA}\\

       \email{kaiser@cs.columbia.edu}\\
}
\fi


\maketitle
\begin{abstract}
  
%Existing Testing and Verification technologies, are impractical for testing large scale softwares. 
%One of the biggest problems faced by developers debugging large scale systems is replicating the deployed environment to figure out errors.
%Additionally, most modern companies are combining their development and operation management activities (DevOps), which has led to an increased emphasis on fast bug resolution.
%In recent years there has been a lot of work in record-and-replay systems which captures traces from live production systems, and replays them.
%However, most such record-replay systems have a high recording overhead and are still not practical to be used in production environments without paying a penalty in terms of user-experience. 
In this work we present a framework which allows users to debug a target production system (execution tracing, profile, breakpoint etc.) in a sandbox environment cloned from the live running system at any point in its execution.
% of a service oriented application. 
The paper leverages user-space containers (OpenVZ/LXC) to launch a container cloned and migrated from running instances of an application, thereby launching two containers: production (which provides the real output), and debug-container (for debugging). 
This \emph{debug-container} provides a sandbox environment for safe application of instrumentation tools without any perturbation to the actual production environment. 
%Test cases are initiated using user-defined probe points which launch test-cases using the execution context of the probe point. 
%Our sandboxes provide name-space, and resource management for the processes executing the test/debug cases.
A customized-network proxy agent replicates inputs from clients to both the production and debug-container, as well as safely discards all outputs from the debug-container.
%These sandboxed test-containers can be run either on the same physical host as the production container, or scaled out on dedicated physical test servers.
%, and manage the file system such that existing and newly created file descriptors are safely managed.
%Further we explain fidelity guarantees of our proposed system
We believe our tool provides a novel mechanism for practical live-debugging of large scale multi-tier and cloud applications, without requiring any application down-time, and minimal performance impact.
%In our evaluation provide a number of use-cases to show the utility of our tool.  
\end{abstract}


% A category with the (minimum) three required fields
\category{H.4}{Information Systems Applications}{Miscellaneous}
%A category including the fourth, optional field follows...
\category{D.2.8}{Software Engineering}{Metrics}[complexity measures, performance measures]

\terms{Theory}

\keywords{ACM proceedings, \LaTeX, text tagging} % NOT required for Proceedings

\bibliographystyle{abbrv}
\bibliography{sigproc}  % sigproc.bib is the name of the Bibliography in this case
% You must have a proper ".bib" file
%  and remember to run:
% latex bibtex latex latex
% to resolve all references
\end{document}
