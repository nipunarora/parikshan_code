\section{Introduction}
\label{sec:intro}

Modern 24x7 software systems are often shipped out with undetected bugs despite the best efforts of programmers and bug detection tools. 
Fixing these bugs often requires painstaking effort in trying to recreate the problem, isolate and localize the root-cause of the error.
In a large-scale distributed, the process of recreating an error can be tedious, lengthy as well as costly in terms of the infrastructure required.
The time taken in debugging such errors can lead to a substantial loss of revenue.
Hence, it is important to have a fast and easy environment to recreate problems, with the shortest possible time to bug resolution.

The live debugging~\cite{parikshan} project aims to streamline the process of debugging by reducing the gap between the production and the debugging environment. 
It achieves this goal by leveraging user-space container virtualization, and live migration technologies in order to create a replica debug environment. 
The framework clones active production systems(production-containers), and creates a debug-container, which receives the same input as the original production container.
Hence both the production and debug-container, are running in parallel in the same state, while only the output of the production container is visible to the user.
This clear separation of the production and debug environment, significantly reduces time to bug resolution by allowing developers to safely debug the application in the debug-container, without impacting user-facing operations.

However, one of the limitations of live debugging is the need to ensure that the time-lag induced in the debug container because of instrumentation,  does not exceed a preset threshold which stops the containers from syncing their state.
Essentially, the debug container maintains a buffer where it queues the incoming requests, 

In this work, we model the 
