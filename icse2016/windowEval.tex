\begin{table}[ht]
	\centering
	\begin{tabular}{|c|r|r|r|}
		\hline
		{\bf \begin{tabular}[c]{@{}c@{}}Input \\ Rate\end{tabular}} & \multicolumn{1}{c|}{{\bf \begin{tabular}[c]{@{}c@{}}Debug \\ Window\end{tabular}}} & \multicolumn{1}{c|}{{\bf \begin{tabular}[c]{@{}c@{}}Pipe \\ Size\end{tabular}}} & \multicolumn{1}{c|}{{\bf \begin{tabular}[c]{@{}c@{}}Slow- \\ down\end{tabular}}} \\ \hline
		530 bps, 27 rq/s                                            & $\infinity$                                                                                     & 4096                                                                            & 1.8x                                                                             \\ \hline
		530 bps, 27 rq/s                                            & 8 sec                                                                              & 4096                                                                            & 3x                                                                               \\ \hline
		530 bps, 27 rq/s                                            & 72 sec                                                                             & 16384                                                                           & 3x                                                                               \\ \hline
		Poisson, λ = 17 rq/s                                        & 16 sec                                                                             & 4096                                                                            & 8x                                                                               \\ \hline
		Poisson, λ = 17 rq/s                                        & 18 sec                                                                             & 4096                                                                            & 5x                                                                               \\ \hline
		Poisson, λ = 17 rq/s                                        & $\infinity$                                                                                     & 65536                                                                           & 3.2x                                                                             \\ \hline
		Poisson, λ = 17 rq/s                                        & 376 sec                                                                            & 16384                                                                           & 3.2x                                                                             \\ \hline
	\end{tabular}
\captionsetup{justification=centering}
\caption{Approximate debug window sizes for a MySQL request workload}
\label{table:timewindow}
\end{table}

\iffalse
\begin{table}[ht]
\begin{centering}
\begin{tabular}{|c|c|c|c|}
\hline
\begin{tabular}[c]{@{}c@{}}\textbf{Input} \\ \textbf{Rate}\end{tabular} & \begin{tabular}[c]{@{}c@{}}\textbf{Debug}\\ \textbf{Window}\end{tabular} & \begin{tabular}[c]{@{}c@{}}\textbf{Pipe}\\ \textbf{Size}\end{tabular} & \begin{tabular}[c]{@{}c@{}}\textbf{Slow-}\\ \textbf{down}\end{tabular}\\ \hline
530 bps, 27 rq/s                                      & $\infinity$                                                    & 4096                                                & 1.8x                      \\ \hline
530 bps, 27 rq/s                                      & 8 sec                                                  & 4096                                                & 3x                        \\ \hline
530 bps, 27 rq/s                                      & 72 sec                                                 & 16384                                               & 3x                        \\ \hline
Poisson, $\lambda$ = 17 rq/s                               & 16 sec                                                 & 4096                                                & 8x                        \\ \hline
Poisson, $\lambda$ = 17 rq/s                               & 18 sec                                                 & 4096                                                & 5x                      \\ \hline
Poisson, $\lambda$ = 17 rq/s                               & $\infinity$                                                    & 65536                                               & 3.2x \\ \hline
Poisson, $\lambda$ = 17 rq/s                               & 376 sec                                                & 16384                                               & 3.2x \\ \hline
\end{tabular}
\captionsetup{justification=centering}
\caption{Approximate debug window sizes for a MySQL request workload}
\label{table:timewindow}
\end{centering}
\end{table}
\fi
	


\subsection{Debug Window Size}
\label{sec:timewindowPerformance}

\noindent
We call the time taken to reach a buffer overflow the ``debug-window'' for the debug-container.
As explained earlier (see section \ref{sec:timewindowPerformance}), the size of this debug-window, depends both on the overhead of the ``instrumentation'', the incoming workload distribution, and the size of the buffer.
To evaluate the approximate size of the debug-window, we sent requests to a production and debug MySQL container via our network duplicator.
Each workload ran for about 7 minutes (10,000 ``select * from table'' queries), with varying request workloads.
We also profiled the server, and found that is able to process a max of 27 req/s\footnote{Not the same as bandwidth, 27 req/s is the maximum rate of sequential requests MySQL server is able to handle for a user session} in a single user connect session. 
For each of our experiments we vary the buffer sizes to get an idea of debug-window. 
Additionally, we generated a slowdown by first modeling the time taken by MySQL to process requests (27 req/s or 17req/s), and then putting an approximate sleep in the request handler.

Initially, we created a connection, and sent requests at the maximum request rate the server was able to handle (27 req/s).
We found that for overheads up-to 1.8x (approx) we experienced no buffer overflows.
% most function tracing and profiling tools generally have an overhead of 1-1.5x for lower granularity and 2x for higher granularity tracing.
For higher overheads the debug window had a linear increase primarily based on buffer-size, request size, and slowdown.

Next, we mimic user behavior, to generate a realistic workload.
We simulate a poisson process with an average request rate of 17 requests per second. 
This varies the inter-request arrival time, and let's the cloned debug-container catch up with the production container during idle time-periods in between request bursts.
We observed, that with the default buffer size (linux default pipe size is 65536), the system was able to tolerate a much higher overhead (3.2x) with no buffer overflows.



\noindent\fbox{%
	\parbox{\columnwidth}{%
		To answer \textbf{RQ2}, we found that the debug window size depends on the incoming workload, the slowdown caused due to instrumentation overhead, and the size of the buffer. We conclude that with a large enough buffer size, most applications will have a sufficiently long debug-window (several minutes), to allow them to capture bug traces, and find the error.
		Buffer overflow will typically happen only for very large spikes in a connected session from the user.
	}%
}

%The buffer size is configurable and for most of our case studies we kept it at 1MB.
%Hence, this is a configuration parameter that depends on the desired window size and the workload.
%We also tried a gaussian request distribution with request interarrival time being randomized to exhibhit a bursty behavior with high utilization. 
%We found that 
 
%In this section we evaluate the testing window size using varying amounts of instrumentation, and the workload.
%For the purpose of this evaluation, we keep a fixed buffer size. 
%First we use a controlled workload rate, and gradually increase the overhead, then we use another scenario, where we keep the try to keep the same overhead, and try to increase the workload.
%We also use real-world network packet capture data, to simulate a realistic workload and gradually increase the overhead there

%\texttt{Nipun's note: This section still needs to be completed, I'm finishing up some of the results, before I can generate the charts}
