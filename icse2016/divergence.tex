
\subsection{Divergence Checking}
\label{sec:divergenceChecking}

%\textbf{Divergence checking}: 
%As mentioned earlier, we clone the entire state of the production container, and replicate the incoming requests. 
%We believe that this should capture most non-determinism in the application, and the debug container should be a close representative of the production container. 
%However, 
%\noindent
Record-and-replay systems capture all possible sources of non-determinism, which ensures that they replay the exact same trace. 
However, in \parikshan it is possible that non-deterministic behavior in the two containers, or modifications because of user instrumentation, causes the production and debug container to diverge with time.
To understand and capture this divergence, we compare the corresponding network output received in the proxy.
This is an optional component, which gives us a black-box mechanism to check the fidelity of the debug-container based on it's communication with external components.
Comparisons use hash of data packets, which are collected and stored in memory for the duration that the connections are active.
The degree of acceptable divergence (the point till which the production and debug containers can be assumed to be in sync) is dependent on the application behavior, and the operator. 
For example, an application which is sending timestamps in each of it's send messages can be expected to have a much higher degree of divergence, in comparison to an application which is querying a database.