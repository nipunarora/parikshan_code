\section{Application}
\label{statistical}

\textbf{Statistical Testing:}
One well known technique for debugging production application is statistical debugging. 
This is achieved by having predicate profiles from both successful and failing runs of a program and applying statistical techniques to pinpoint the cause of the failure.
The core advantage of statistical debugging is that the sampling frequency of the instrumentation can be decreased to reduce the instrumentation overhead.
Despite it's advantages the instrumentation frequency for such debugging to be successful needs to be statistically significant. 
Furthermore, unlike \parikshan, statistical debugging will impose an overhead on the actual application. 

We believe, that statistical debugging can be successfully applied in the debug-container when an error has been observed. 
The entire scope of such a live debugging mechanism is out of scope of this paper, however for completeness sakes we have briefly described it's application here. 
\parikshan can complement statistical debugging in several ways to make it more effective, while at the same time isolating the instrumentation impact on the production container.
Firstly, using dynamic instrumentation tools \parikshan can instrument as well as modify the predicates that are instrumented.
Further the instrumentation can be increased or decreased dynamically, by taking the amount of buffer currently utilized. 
The buffer utilization is an indication of how much the debug-container lags behind the production container, and how long it would take for the buffer to overflow.

\textbf{Record and Replay:}

\textbf{}