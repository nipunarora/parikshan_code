\section{Conclusion \& Future Work}
\label{sec:conclusion}

%Quick bug resolution is an important requirement for any production system maintenance effort.
%Existing techniques either add substantial overhead making them impractical in real-world systems, or are unable to capture enough trace information to localize the bug offline. 

%We presented \parikshan a framework to do live-debugging and analysis of large scale systems.
\parikshan is a novel framework to debug systems in real-time allowing for fast bug resolution.
We have shown the applicability of \parikshan in  7 real-world case studies, and shown how it can be used to quickly localize the error.

In the future, we wish to explore \parikshan further in 3 key areas: (1). Application: we aim to apply our system to real-time intrusion detection, and statistical debugging.
(2). Analysis: we wish to define ``real-time'' data analysis techniques for traces and instrumentation done in \parikshan.
%We believe with streaming data analytic techniques, we can do much better than execution tracing/profiling.
(3). Optimize Live Cloning: we plan to reduce the suspend time of ``live cloning'', by applying optimizations suggested in several recent works in live migration.
We also implemented a prototype of \parikshan on Google's Cloud Infrastructure (Google Compute~\cite{gcompute}), we plan to use this prototype in future publications for scaled out testing. 
%Running a clone in the same machine as the production container opens several venues such as copy-on-write migration, and asynchronous container syncing etc.

An extended version of the paper is present in CUCS Tech Reports~\cite{parikshanTR,parikshan}, and all source code is available on github~\cite{github} 
Kaiser is funded in part by NSF CCF-1302269 and CCF-1161079
%comment{
%We would like to acknowledge Qiang Xu, Abhishek Sharma, and Pallavi Joshi for their insight and feedback in designing \parikshan, and in the evaluation of this technology.
%The authors are affiliated with NEC Labs, Princeton, Google Inc, and Columbia University. 
