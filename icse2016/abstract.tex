
%SIGMETRICS
\begin{abstract}
\noindent
%One of the biggest problems faced by developers debugging large scale systems is replicating the deployed environment to figure out errors. 
%Additionally, many modern companies are combining their development and operation management activities (DevOps), which has led to an increased emphasis on fast bug resolution. 
Modern 24x7 systems rely on short deployment cycles, and fast bug resolution to maintain their services. 
Hence, time-to-bug localization is extremely important for any SOA application.
In this work, we present \textit{``live-debugging''}, a mechanism which allows debugging of production systems (run test-cases, debug, or profile, etc.) on-the-fly. 
%in a sandbox environment cloned from the live running system. 
We leverage user-space virtualization technology (OpenVZ/LXC), to launch containers cloned and migrated from running instances of an application, thereby having two containers: \textit{production} (which provides the real output), and \textit{debug} (for debugging). 
The \textit{debug-container} provides a sandbox environment for debugging without any perturbation to the production environment. 
%Our sandboxes provide name-space, and resource management for the processes executing the test/debug cases. 
Customized network-proxy agents replicate or replay network inputs from clients to both the production and debug-container, as well as safely discard all network output from the debug-container. 
%The sandboxed debug-container can be run either on the same physical host as the production container, or scaled out on dedicated physical debug servers. 
We used our system, called \texttt{Parikshan}, to do \textit{``live-debugging''} on several real-world bugs, and effectively reduced debugging complexity and time.
%We believe our tool provides a mechanism for practical live-debugging of large scale multi-tier and cloud applications, without requiring any application down-time, and minimal performance impact.

\noindent
\textbf{Tags:} live cloning, network duplication, live debugging
\end{abstract}

%In recent years there has been a lot of work in record-and-replay systems which captures traces from live production systems, and replays them.
%However, most such record-replay systems have a high recording overhead and are still not practical to be used in production environments without paying a penalty in terms of user-experience. 
