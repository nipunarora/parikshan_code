\section{Discussion and Challenges}
\label{sec:challenges}

In this section we describe some of the challenges, and limitations 

\subsection{Non-Determinism}
\label{sec:nonDeterminism}

Execution of the same input in different conditions can produce different outputs from an application, this behavior is referred to as non-deterministic execution.
With the exception of dedicated controlled executions in embedded systems etc., most large scale applications and complex systems exhibit non-determinism, which is what makes debugging a challenge.
Broadly non-determinism can be caused because of the following reasons: 1. Input non-determinism, 2. Configuration non-determinism, 3. Concurrency based non-determinism.

Input non-determinism is non-deterministic behavior caused because of different input received by the application (eg. requests to the server, user input etc.). 
Obviously, any output in the application and it's execution trace, will depend on the input it receives. 
In offline debugging it is often difficult to capture all possible inputs, and hence deal with input non-determinism.
In order to avoid input non-determinism, \parikshan uses a network proxy to send all inputs received in the production container also to the test-container. 
Hence both of them behave in almost the same fashion.

Configuration non-determinism relates to the state of the container, the application configuration, system/kernel parameters, hardware resources etc.
Once again, this can often effect the logic of the application.
For eg. max no. of theads is a common configuration parameter in several multi-process server applications, and can effect the performance of the application in high workloads.
\parikshan can deal with most configuration/state non-determinism as it captures the state of the production container at the time of cloning, and creates an exact replica.
This replica should have an identical behavior to the original and should have similar memory consumption, threading behavior etc. 
However, some amount of inconsistency between the test-container and production container is possible.

Another kind of non-determinism is caused by concurrent applications or parallel applications which may behave differently in each execution.
Capturing this kind of non-determinism is important to debug bugs such as race-conditions, locks etc.
Unfortunately, the current system design of \parikshan cannot capture concurrency based non-determinism, we hope to revise and include this in future versions.

\subsection{Slowdown}
\label{sec:slowdown}


\subsection{Consistency Requirements}
\label{sec:consistency}
