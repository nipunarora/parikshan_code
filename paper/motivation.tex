
\section{Motivation}

\subsection{Motivationg Scenario}

Take for example user Joe who is an administrator, and IT manager for a multi-tiered system. 
Much like several IT systems user Joe has a dashboard which informs him of the health status of all of his applications, and provides him with high level statistical views of all tiers of the system.
At time t0, Joe observes an unusually high memory usage by tierA for transaction type X or unusually high latencies in fetch operations for user Y.
Under usual circumstances, the system would have to go down(depending on the severity of the problem), a ticket would be generated for the developer and the system would be patched once the problem has been diagnosed.
However often, it is difficult to find out the confiugration of the system, and the user input which is causing this problem, also solving any emergent problems as soon as possible is extremely important.

Joe can now use \textit{Parikshan}, to fork off a clone of tierA as test-tierA. Our proxy balancer sends a copy of the incoming request to test-tierA, while users can continue using tierA. 
Process in test-tierA follow the same execution paths, as they receive the same input(we discuss non-determinisim related issues later); this allows Joe to initiate deeper test-cases, and observe the test-tierA, without fearing any problems in the user-facing operations.

Time to bug resolution is usually a very important criteria in any user-facing service oriented application.
Bearing this in my mind we believe, that online testing will be an important aspect towards modern applications.
Additionally the usage of redundant computing for testing in alpha-beta testing(see \ref{sec:related}) approaches is a well accepted paradigm in real-world applications.
This leads us to believe that using redundant computing should be acceptable for regular testing approaches as well.

\subsection{Motivation Questions?}

To further motivate our testing paradigm we have come up with a set of motivating questions:

\begin{compactitem}
\setlength{\itemsep}{1Pt}
\item[]\textbf{Q1:} Is it important to sandbox test-cases?
\item[]\textbf{Q2:} Is recreating production environment difficult? 
\item[]\textbf{Q3:} Is redundant computing available? 
\item[]\textbf{Q4:} How would executing test-cases in a production server effect user-experience?
\end{compactitem}

\subsubsection{\textbf{Q1:} Is it persistent testing important?}
\subsubsection{\textbf{Q2:} Is recreating production environment difficult?}
\subsubsection{\textbf{Q3:} Can redundant computing be utilized for testing?}
\subsubsection{\textbf{Q4:} How would executing test-cases in a production server effect user-experience?}
