%SIGMETRICS
\begin{abstract}
One of the biggest problems faced by developers debugging large scale systems is replicating the deployed environment to figure out errors.
Additionally, most modern companies are combining their development and operation management activities (DevOps), which has led to an increased emphasis on fast bug resolution.
In this work we present a framework which allows users to debug the target production system (run test-cases, debug, or profile etc.) in a sandbox environment cloned from the live running system at any point in it's execution.
The paper leverages user-space containers (OpenVZ/ LXC) to launch a container cloned and migrated from running instances of an application, thereby launching two containers: production (which provides the real output), and test-container (for debugging). 
This \emph{test-container} provides a sandbox environment, for safe execution of test-cases/debugging done by the users without any perturbation to the execution environment. 
Our sandboxes provide name-space, and resource management for the processes executing the test/debug cases.
A customized-network proxy agent replicates inputs from clients to both the production and test-container, as well as safely discard all outputs from the debug-container.
These sandbox debug-containers can be run either on the same physical host as the production container, or scaled out on dedicated physical test servers.
We believe our tool provides a mechanism for practical live-debugging of large scale multi-tier and cloud applications, without requiring any application down-time, and minimal performance impact.

\textbf{Tags:} software debugging, live cloning, network proxy, live debugging
\end{abstract}

%In recent years there has been a lot of work in record-and-replay systems which captures traces from live production systems, and replays them.
%However, most such record-replay systems have a high recording overhead and are still not practical to be used in production environments without paying a penalty in terms of user-experience. 
