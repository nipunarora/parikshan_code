\begin{abstract}
  
%Existing Testing and Verification technologies, are impractical for testing large scale softwares. 
One of the biggest problems faced by developers debugging large scale systems is replicating the deployed environment to figure out errors.
In recent years there has been a lot of work in record-and-replay systems which captures traces from live production systems, and replays them.
However, most such record-replay systems have a high recording overhead and are still not practical to be used in production environments without paying a penalty in terms of user-experience. 

In this work we present a harness for production systems which allows users to debug the target system (run test-cases, debug, or profile etc.) in a sandbox environment cloned from a live running system at any point in it's execution.
% of a service oriented application. 
The paper leverages, user-space containers (OpenVZ/ LXCs) to launch a container cloned and migrated from a running instances of an application, thereby launching two containers: production (which provides the real output), test-container (for debugging/testing). 
This \emph{test-container} provides a sandbox environment, for safe execution of test-cases/debugging done by the users without any perturbation to the execution environment. 
%Test cases are initiated using user-defined probe points which launch test-cases using the execution context of the probe point. 
Our sandboxes provide a have namespace, and resource management for the processes executing the test/debug cases.
A customized-network proxy agent replicates inputs from clients to both the production and test-container, as well safely discard all outputs from the test-container.
%, and manage the file system such that existing and newly created file descriptors are safely managed.
%Further we explain fidelity guarantees of our proposed system
We believe our tool provides a mechanism for practical live-debugging of large scale multi-tier and cloud applications, without requiring any application down-time, and minimal performance impact.
%In our evaluation provide a number of use-cases to show the utility of our tool.  
\end{abstract}
