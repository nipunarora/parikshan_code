\noindent\section{Inserting probes \& Performance Analysis}
\label{sec:trigger}

The key idea behind sandbox testing is to debug problems in real-world scenarios
Once we have forked off a clone, we are now ready to do some deeper analysis. 
Here we explain how debugging can potentially be done in real live systems to give the operator and developers a deeper view of the target application.
\parikshan could potentially be used in several ways for application debugging and introspection. 
However, in the interest of brevity, we talk about only two categories. 
Firstly, inserting probes which can help in generating traces so that the user can understand the execution path of the application.
Secondly, performance analysis, such as memory usage, number of context switches etc. this can give the user an idea of what could be the bottleneck for the target application.

%By design, \textit{Parikshan} can allow a variety of test cases. 
%We divide such analysis in two parts based on the time window required for analysis: (1).Statistical

\noindent\subsection{Inserting Probes \& TestCases}
\label{sec:unitTests}

Execution tracing is a key debugging tool for locating any bugs. 
Essentially, an execution trace is a log of the functions executed, values of certain variables being tracked etc. within a time-frame.
The path taken and the value of the variables are important as they explain the logic flow of the application for the \emph{"configuration"} and \emph{"context"} at the time of creating the clone.
To generate this trace, we insert probes or watch points for variables in the application similar to existing development phase debugging tools such as GDB(gnu debugger) \cite{gdb}.
Inserting probes in the sandbox can be done using any existing dynamic instrumentation tools  such as PIN \cite{pin}, Valgrind \cite{valgrind}, Dyninst \cite{dyninst}.
For our purposes we used \iprobe\footnote{we have used iProbe only because of our familiarity} \cite{iProbe} a dynamic binary instrumentation tool designed by the authors to insert probes in running applications. 
 \iprobe  uses compile time flags to generate placeholders in the binary, and inserts instrumentation probes in the binary at run-time.
Users can use it to insert print statements on function executions. 



\subsection{Performance Analysis}
\label{sec:Performance}

While execution traces provide a good indication for localizing bugs, often system level statistics, such as memory usage, cpu usage, context switches, number of threads can be a good indication towards bug localization.
This is especially true if the bug is because of a configuration error or in the interaction of the application to the base kernel system.

Given that the production and the test-container have similar resources allocated to them, we believe that the system level statistics of the test-container will be a close representation of the original as it was a clone of the production container.
We accept that it cannot represent the production container entirely, as there is bound to be some non-determinism in the containers.
However, any input driven state changes are bound to be caught.

Previous approaches including some of the work done by the authors \cite{clue} \cite{kscope} and others in academia \cite{sherlog} \cite{vscope} have often used kernel level statistics or system call traces, in localizing bugs. 
The idea behind these approaches is generally to provide a black-box solution to applications.
However, in practice they can only be used in lower workload's or can only provide vague clues as they need to restrict the amount of instrumentation to avoid a heavy overhead.

%Analysis which need a long time window to record, and run the status across multiple requests are considered as long running analysis. 
%Such analysis can be considered to be similar to monitoring of live applications, and are usually statistical in nature.
%Typically tools such as PIN \cite{pin}, Valgrind \cite{valgrind}, Dyninst \cite{dyninst}, can do deep analysis without modifying the logic of the application.
%However, they impose a heavy penalty in terms of performance.
%Such tools can be easily used in \textit{Parikshan}, without effecting system performance.
%However, there are a few challanges with such statistics which need a longer window to run.

