\section{Triggering and Inserting Analysis}
\label{sec:trigger}

The key idea behind sandbox testing is to debug problems in real-world scenarios
Once we have forked off a clone, we are now ready to do some deeper analysis. 
%By design, \textit{Parikshan} can allow a variety of test cases. 
We divide such analysis in two parts based on the time window required for analysis: (1).Statistical

\subsection{Inserting Probes \& TestCases}
\label{sec:unitTests}

Inserting probes in the sandbox can be done using existing dynamic instrumentation

\subsection{Statistical Analysis}
\label{sec:statisticalTests}

Analysis which need a long time window to record, and run the status across multiple requests are considered as long running analysis. 
Such analysis can be considered to be similar to monitoring of live applications, and are usually statistical in nature.
Typically tools such as PIN \cite{pin}, Valgrind \cite{valgrind}, Dyninst \cite{dyninst}, can do deep analysis without modifying the logic of the application.
However, they impose a heavy penalty in terms of performance.
Such tools can be easily used in \textit{Parikshan}, without effecting system performance.
However, there are a few challanges with such statistics which need a longer window to run.

