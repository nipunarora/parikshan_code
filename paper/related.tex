\section{Related Work}
\label{sec:related}

%There have been several existing approaches that look into testing applications in the wild. 
%The related work can be divided in several categories:

%\begin{itemize}[leftmargin=*]
\iffalse  
%\item 
\textbf{Software Debugging}  
In the development phase, it is common to employ debugging tools such as gnu debugger \cite{gdb}, valgrind \cite{valgrind} or just using print statements etc.
Several development suites\cite{eclipse, visual_studio, intel_suite} often come with inbuilt debugging capabilities, to assist developers to understand their code, and debug it as they develop new applications.
In most cases these tools allow developers to look at the execution traces, and to insert watchpoints or breakpoints.
In addition they allow developers to understand the context of the application by looking at variable values at different points.
Unfortunately, this practice cannot be followed in production environments, as these tools have a high overhead.  
\parikshan focuses on this problem by allowing users to do live debugging of the application by cloning the production state, to produce a test-container.
This test-container can be debugged using probes as described in~\ref{sec:trigger} to give valuable insight to the developer.
\fi
    
%\item 
\noindent
\textbf{Record and Replay:}  
 Record and Replay systems~\cite{odr,revirt,guo2008r2, geels2007friday, laadan2010transparent} have been an active area of research in the academic community for several years. 
Typically these systems record enough information to faithfully replay the system offline.
However, the overhead of most record-and-replay systems are quite high, and recording is generally done in isolation for each VM.  
In comparison to record-and-replay systems, \parikshan allows debuggers to diagnose the bug in real-time.
Another closely related work is Aftersight~\cite{aftersight} which records a production system, and replays it concurrently in a parallel VM.
While both Aftersight and \parikshan allow debuggers an almost real-time diagnosis facility, Aftersight suffers from recording overhead in the production VM.
On the other hand, in \parikshan we do not impact the production container, as we only duplicate the incoming traffic instead of recording.

\noindent
\textbf{Large Scale System Debugging}
Production level diagnosis~\cite{magpie,clue,vpath} approaches use light weight instrumentation to capture traces like system calls, or windows event traces.
Both CLUE and VPath use end-to-end trace stitching to capture the execution flows across multiple tiers to infer performance bugs.
While the instrumentation for these tools have a low overhead, the corresponding granularity of the logs is also less.
This limits the diagnosis capability of these tools, and they are only able to give a hint towards the bug root-cause rather than debug it completely.
Since, \parikshan does debugging on cloned containers, it can do intensive instrumentation making the diagnosis much easier.

\noindent
\textbf{Real-Time techniques}
In recent years there have been approaches which are similar to \parikshan in applying real-time diagnosis on production systems.
One of these approaches is Chaos Monkey-\cite{chaosmonkey} from netflix which uses fault injection in real production systems to test their fault tolerance.
ChaosMonkey induces time-outs, extra resoure usage etc. to inject faults in a running system. 
This is usually done at low usage times.
Another approach called AB Testing~\cite{abtesting}, probabilistically tests updates or beta releases on some percentage of users, while letting the majority of the application users work on the original system.
AB Testing allows the developer to understand user-response to any new additions to the software, while most users get the same software.
While both these approaches focus on real-time-diagnosis on production systems, they are restricted to software testing.
\parikshan on the other hand, allows for interactive analysis to understand the execution flow.
%We are inspired by the notion of perpetual testing\cite{perpetual} which advocates that software testing should be key part of the deployment phase and not just restricted to the development phase.
  
  
 % \item \textbf{A-B Testing}
 % \item \textbf{Symbian Monkey}
  % \item \textbf{DevOps}
%\end{itemize}
  