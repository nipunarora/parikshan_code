\section{Conclusion}
\label{sec:conclusion}

\noindent
Quick bug resolution is an important requirement for any production system maintenance effort.
Existing techniques either add substantial overhead detering their application in real-world systems, or are unable to capture enough trace information to easily localize the bug offline. 

\noindent
We have presented \parikshan a mechanism to do live-debugging and analysis of large scale systems.
\parikshan is a novel technique to debug live systems allowing faster times to reach bug resolution.
We evaluated \parikshan on real-world scenarios to show it's applicability \& discussed several modes in which it can be used.
%\comment{
%We would like to acknowledge Qiang Xu, Abhishek Sharma, and Pallavi Joshi for their insight and feedback in designing \parikshan, and in the evaluation of this technology.

The authors are affiliated with NEC Labs, Princeton, Google Inc, and Columbia University. 
Kaiser is funded in part by NSF CCF-1302269, CCF-1161079, and NIH U54 CA121852.
%}
