\section{Evaluation}
\label{sec:evaluation}
In this section we present the evaluation of Parikshan. 
The key questions facing us were:
\begin{itemize}
  \item How can \parikshan be used in the real world? 
  \item Does the test container faithfully represent the execution and the state of the production container? 
   \item How does cloning the container effect the performance of the production container?
   \item How does running tests in the test-container effect the performance of the production container? 
\end{itemize}

\subsection{CaseStudies}
\label{sec:casestudy}

Parikshan enables the users to freely run any test-case in the test-container while not effecting the production container. 
At the same time the output of these tests should not effect the functionality or the performance of the production system.
The main advantage of such a system can be seen in service oriented applications which are user facing and can hence ill-afford to be shutdown for inspecting bugs.
As mentioned earlier, another major advantage is that we are able to capture live user-input. \\

\par \noindent \textbf{CaseStudy 1: Debugging using Execution Tracing of MySQL bug 18511}: \\ 
Performance profiling such as function execution trace, execution time, resource usage etc., is often used to indicate and localize performance bugs in real world systems. 
While performance profiling is simple to implement, it obviously incurs an overhead and will effect user-experience of the target system.
Effectively, this means that despite it's advantages, the amount of profiling that can be done in a production system is extremely limited. 

As our first case study we focused on profiling and capturing a performance bug in a session with several randomly created user transactions to MySQL. 
It was observed, that some of the user requests were running significantly slower than others. 
Hence 

\par \noindent \textbf{CaseStudy 2: Performance Profiling, using dyninst}
%Demand Driven Testing - https://www.cs.virginia.edu/~soffa/Soffa_Pubs_all/Conferences/Demand-driven.Misurda.2005.pdf 

%@Nipun- I don't think either of the next two make a lot of sense
\par \noindent \textbf{CaseStudy 3: A/B Testing}
A/B Testing 
%http://www.polteq.com/wp-content/uploads/2013/02/testingexperience20_12_12_Marc_van_t_Veer.pdf

\par \noindent \textbf{CaseStudy 4: Fault Tolerance Testing}
Fault Injection to look at fault tolerance
%Fault Injection in netflix http://techblog.netflix.com/2012/07/chaos-monkey-released-into-wild.html
Security Honeypots? 

\subsection{Performance}
\label{sec:performance}

In order to evaluate the performance of sandbox testing, we seperated our evaluation in looking at two different stages
\subsubsection{Micro-benchmarks}
\label{sec:micro}

