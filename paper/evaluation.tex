\section{Evaluation}
\label{sec:evaluation}
In this section we present the evaluation of Parikshan. 
The key questions facing us were:
\begin{itemize}
%  \item How can \parikshan be used in the real world? 
%  \item Does the test container faithfully represent the execution and the state of the production container? 
   \item How does cloning the container effect the performance of the production container?
   \item How does running tests in the test-container effect the performance of the production container?
   \item How long of a testing-window do we have? 
\end{itemize}

In order to answer these questions, we seperated our evaluation in looking at two different stages: cloning stage, time-window analysis.

\subsection{Cloning: Micro-Benchmarks}
\label{sec:performance}

\begin{table*}[ht]
  \centering
    \begin{tabular}{ | p{4cm} | l | l | l | l | l | l | l | l | l |}
    \hline
    \textbf{Modes} & \multicolumn{3}{|c|}{\textbf{Internal Mode}} & \multicolumn{3}{|c|}{\textbf{External Mode}} & \multicolumn{3}{|c|}{\textbf{Google Compute}}\\\hline
    \textbf{ } & \textbf{Cl} & \textbf{Hog} & \textbf{Hog+Cl} & \textbf{Cl} & \textbf{Hog} & \textbf{Hog+Cl} & \textbf{Cl} & \textbf{Hog} & \textbf{Hog+Cl} \\ \hline
    \hline
    \textbf{Throughput} & -- & 1691.0 req/s & 1509 req/s & -- & 712 & 625 & -- & 510 & 450\\ \hline
    \hline
    \textbf{Suspend + Dump} & 0.00 & -- & 0.00 & 0.00 & 0.00 & 0.00 & 0.00 & 0.00 & 0.00\\ \hline
    \textbf{Pcopy after suspend} &  -- & 0.00 & 0.00 & 0.00 & 0.00 & 0.00 & 0.00 & 0.00 & 0.00\\ \hline
    \textbf{Copy Dump File} & 0.00 &  -- & 0.57 & 0.74 & 0.59 & 0.65 & 0.00 & 0.00 & 0.00\\ \hline
    \textbf{Undump and Resume} & 0.00 &  -- & 0.57 & 0.74 & 0.59 & 0.65 & 0.00 & 0.00 & 0.00\\ \hline 
    %\textbf{--------------} & --- & --- & --- & --- & --- & --- & --- & --- & --- \\ 
    \hline
    \textbf{Total Suspend Time} & 0.60 &  -- & 0.57 & 0.74 & 0.59 & 0.65 & 0.00 & 0.00 & 0.00\\ \hline
    \end{tabular}
\caption{Performance of Live Cloning (external mode) with a random file dump process running in the container}
\label{table:clonePerf}
\end{table*}


We first looked at the performance of the cloning operation to look at the time taken to do cloning while. In figure \ref {fig: clonePerf}, the first column gives the average performance of 


As explained earlier, the cloning process can be divided into two parts: an rsync operation which does an ``pre-copy'' of the VM, and a followup rsync operation while the target container is suspended, to make sure that both the production and test containers have the exact same state.
The idea is to reduce the time taken to suspend the production container, so that it has minimal effect on the user.
The main factor that effects this is the number of ``dirty pages'' in the suspend phase, which have not copied over in the pre-commit rsync operation.
Natrually, the number of write operations in the container while cloning the container, will increase the number of dirty pages, and increase the time of the suspend operation.


