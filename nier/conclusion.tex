\section{Conclusion \& Future Directions}
\label{sec:conclusion}

In this paper we described a framework to allow live-debugging of large scale systems.
%% FI \parikshan is a novel technique to debug live systems allowing faster times to reach bug resolution.
We relied on the feasibility of live-cloning as a driver for debugging, and looked at some real-world debugging scenarios.

We believe that this vision opens several interesting research directions: 
(1) How can ``live-debugging'' be leveraged to do ``software testing'' in the production environment?
(2) Can this be made a part of the software release cycle, to test beta-releases in the production environment?
(3) How can developers collaborate to make applications live-debugging friendly?
(4) How can several instances of \parikshan be orchestrated to do debugging in multiple VM's of the same workflow?

We would like to acknowledge Franjo Ivan\v{c}i\'{c} for his valuable feedback in the design of this system.
%We evaluated \parikshan on real-world scenarios to show it's applicability and discussed several modes in which it can be used.

%We would like to acknowledge ... for their insight and feedback in designing \parikshan, and in the evaluation of this technology.
%This authors are affiliated with NEC Labs, Princeton and PSL Lab at Columbia University. 
