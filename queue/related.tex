\section{Related Work}
\label{sec:related}

Queuing theory has been used to solve several real-world problems. 
In particular queuing theory has found uses in supply chain optimization.
For example Fedex, and retailers like best-buy have used queuing theory to optimize their delivery schedules.
In the IT field, it has been used by companies like youtube, to optimize queuing in various transaction oriented service systems.
Similarly several other transaction based service oriented applications, are often used for optimizing, and providing service level agreement guarantees.

One other area, where queuing theory has been frequently applied is for load balancing service applications.
For example a common use-case is traffic being routed from a DNS server to numerous webservers. 
Depending on the traffic load the DNS server acts like a proxy and forwards traffic to the webserver, which is relatively free (round robin scheme, or first in first out etc.).
Queuing theory can help model such traffic forwarding, by suggesting more scaled out system depending on the workload, and the amount of accepted transaction processing time.
In this paper we have extended a similar model towards our proxy buffer capacity modeling and instrumentation, in order to extend the time-window to it's maximum.