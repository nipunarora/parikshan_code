\section{Conclusion \& Future Work}
\label{sec:conclusion}

%Quick bug resolution is an important requirement for any production system maintenance effort.
%Existing techniques either add substantial overhead making them impractical in real-world systems, or are unable to capture enough trace information to localize the bug offline. 

We presented \parikshan a framework to do live-debugging and analysis of large scale systems.
\parikshan presents a novel technique to debug systems in real-time allowing for fast bug resolution.
We show several case-studies to see how our system can be applied on real problems.

In the future, we wish to explore \parikshan further in 3 key areas: (1). Application: we aim to apply our system to real-time intrusion detection, and fault tolerance testing.
(2). Analysis: we wish to define ``real-time'' data analysis techniques for traces and instrumentation done in \parikshan.
We believe with streaming data analytic techniques, we can do much better than execution tracing/profiling.
(3). Live Cloning: we plan to reduce the suspend time of ``live cloning'', by applying optimizations suggested in several recent works in live migration.
%Running a clone in the same machine as the production container opens several venues such as copy-on-write migration, and asynchronous container syncing etc.


%comment{
%We would like to acknowledge Qiang Xu, Abhishek Sharma, and Pallavi Joshi for their insight and feedback in designing \parikshan, and in the evaluation of this technology.
The authors are affiliated with NEC Labs, Princeton, Google Inc, and Columbia University. 
Kaiser is funded in part by NSF CCF-1302269, CCF-1161079, and NIH U54 CA121852.
